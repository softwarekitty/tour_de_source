\section{Discussion}
\label{sec:discussion}

\subsection{Utilizations May Be More Common In Python Library Code Than In User Code}

Although \DTLfetch{data}{key}{percentProjectsUsingRegex}{value}\% of the projects observed had at least one regex usage, only \DTLfetch{data}{key}{percentFilesUsingRegex}{value}\% of the files observed had at least one regex usage.  This indicates that regex usage is usually concentrated in a few files.  One exception to this is trend is the project with the most files containing utilizations: Arianrhod\footnote{\url{https://github.com/Ouroboros/Arianrhod}} which is a Japanese Anime game, mostly written in C\# (over 18K files), but containing 3404 Python files, most of which are source code for various libraries.  Of these library files, 541 (15.9\%) contain at least one utilization.


% to find the largest count for the 'utilizing files' row: select distinct uniqueSourceID, filePath, count(distinct filePath) as ct from RegexCitationMerged group by uniqueSourceID order by ct;

From table~\ref{table:saturation}, we see that on average each project had \DTLfetch{data}{key}{medRFilePerProj}{value} files containing any regex usage, out of an average of \DTLfetch{data}{key}{medFilePerProj}{value} files.  Each of the files that did have a regex usage had an average of \DTLfetch{data}{key}{medRPerFile}{value} regex usages.  Because we scanned \DTLfetch{data}{key}{nProjScanned}{value} projects, we would expect to have seen \DTLfetch{data}{key}{nExpectedUsages}{value} regex usages, which is lower than the actual \DTLfetch{data}{key}{nUsages}{value} usages observed.
\subsubsection{Saturation}



\subsection{DBB subsumes repetition, CCC subsumes character classes}

The DBB feature subsumes all other repetition features.  Consider the equivalences shown in Figure~\ref{fig:DBBequivalences}

\begin{figure}[htb]
\centering
\includegraphics[width=\columnwidth]{../illustrations/DBBequivalences.eps}
\caption{How DBB Subsumes All Other Repetition Features}
\label{fig:DBBequivalences}
\end{figure}

Similarly, the CCC feature subsumes NCCC, RNG, ANY, DEC, NDEC, WSP,NWSP, WRD, NWRD because each of these features is equivalent to a set of characters.  We provide an example of how CCC subsumes DEC, WSP and WRD in Figure~\ref{fig:CCCequivalences} (other equivalences not shown for brevity).

\begin{figure}[htb]
\centering
\includegraphics[width=\columnwidth]{../illustrations/CCCequivalences.eps}
\caption{How CCC Subsumes DEC, WSP and WRD}
\label{fig:CCCequivalences}
\end{figure}

\subsubsection{character classes are important}
In replacing keyword search with an abstracted search, one of the most fundamental abstractions is that one element of a sequence can be one of several characters.  This abstraction is realized in custom character classes.

\subsubsection{default character classes are widely used}

The pattern language for Python and most major regex engines supports a few default character classes (and their negations) which we have described as the features ANY, DEC, WSP, WRD, NDEC, NWSP, NWRD.  Throughout this analysis it was obvious that these default character classes were widely used. Specifically, \verb!\s!, \verb!\d! and \verb!\W! were the top three behavioral clusters (as shown in Table~\ref{table:topNClusters}).  In Table~\ref{table:threeClusterSample} we show the top 10 patterns from these top three clusters.


\begin{table}
\begin{center}
\caption{Top 10 Patterns in Top 3 Clusters (RQ3)}
\label{table:threeClusterSample}

\begin{tabular}
{c|c|c}
\toprule
example & example & example\\
\midrule
\begin{minipage}{0.5in}\begin{verbatim}`\s'\end{verbatim}\end{minipage} & \begin{minipage}{0.5in}\begin{verbatim}`\s'\end{verbatim}\end{minipage} & \begin{minipage}{0.5in}\begin{verbatim}`\s'\end{verbatim}\end{minipage}\\
\midrule
\begin{minipage}{0.5in}\begin{verbatim}`\s'\end{verbatim}\end{minipage} & \begin{minipage}{0.5in}\begin{verbatim}`\s'\end{verbatim}\end{minipage} & \begin{minipage}{0.5in}\begin{verbatim}`\s'\end{verbatim}\end{minipage}\\
\midrule
\begin{minipage}{0.5in}\begin{verbatim}`\s'\end{verbatim}\end{minipage} & \begin{minipage}{0.5in}\begin{verbatim}`\s'\end{verbatim}\end{minipage} & \begin{minipage}{0.5in}\begin{verbatim}`\s'\end{verbatim}\end{minipage}\\
\midrule
\begin{minipage}{0.5in}\begin{verbatim}`\s'\end{verbatim}\end{minipage} & \begin{minipage}{0.5in}\begin{verbatim}`\s'\end{verbatim}\end{minipage} & \begin{minipage}{0.5in}\begin{verbatim}`\s'\end{verbatim}\end{minipage}\\
\midrule
\begin{minipage}{0.5in}\begin{verbatim}`\s'\end{verbatim}\end{minipage} & \begin{minipage}{0.5in}\begin{verbatim}`\s'\end{verbatim}\end{minipage} & \begin{minipage}{0.5in}\begin{verbatim}`\s'\end{verbatim}\end{minipage}\\
\midrule
\begin{minipage}{0.5in}\begin{verbatim}`\s'\end{verbatim}\end{minipage} & \begin{minipage}{0.5in}\begin{verbatim}`\s'\end{verbatim}\end{minipage} & \begin{minipage}{0.5in}\begin{verbatim}`\s'\end{verbatim}\end{minipage}\\
\midrule
\begin{minipage}{0.5in}\begin{verbatim}`\s'\end{verbatim}\end{minipage} & \begin{minipage}{0.5in}\begin{verbatim}`\s'\end{verbatim}\end{minipage} & \begin{minipage}{0.5in}\begin{verbatim}`\s'\end{verbatim}\end{minipage}\\
\midrule
\begin{minipage}{0.5in}\begin{verbatim}`\s'\end{verbatim}\end{minipage} & \begin{minipage}{0.5in}\begin{verbatim}`\s'\end{verbatim}\end{minipage} & \begin{minipage}{0.5in}\begin{verbatim}`\s'\end{verbatim}\end{minipage}\\
\midrule
\begin{minipage}{0.5in}\begin{verbatim}`\s'\end{verbatim}\end{minipage} & \begin{minipage}{0.5in}\begin{verbatim}`\s'\end{verbatim}\end{minipage} & \begin{minipage}{0.5in}\begin{verbatim}`\s'\end{verbatim}\end{minipage}\\
\midrule
\begin{minipage}{0.5in}\begin{verbatim}`\s'\end{verbatim}\end{minipage} & \begin{minipage}{0.5in}\begin{verbatim}`\s'\end{verbatim}\end{minipage} & \begin{minipage}{0.5in}\begin{verbatim}`\s'\end{verbatim}\end{minipage}\\
\bottomrule
\end{tabular}
\end{center}
\end{table}



One surprising result of our clustering is that behaviorally speaking, the negation of the word class was more heavily used (208 projects) than the word class itself (114 projects).  One

The character class of all letters \verb![a-zA-Z]! appears so often that it is an excellent candidate for a new a character class.

The largest cluster using custom character classes (cluster N in Table~\ref{table:topNClusters} whose shortest member is \verb•[^ -~]•) has 44 patterns which create a more permissive word class.  Inspection of the source code of several projects using this pattern indicates that these permissive word classes were typically used when trying to create system-friendly object names, which indicates that there is a demand for a word class that includes more characters (but not dashes, tildes or the first 9 unicode characters).




One obvious character class to consider is hexidecimal characters, and we see the pattern [a-fA-F0-9] appear many times.

For tools that do not support the default character classes, this is a significant obstacle for users trying to test the regexes that they already have.

\subsubsection{use of repetition}


\subsubsection{Anchors matter}
The endpoint anchor features STR and END are the only way specify that an entire line has to match.  Consider the following example, comparing the pattern \verb!`^\s*$'! (found in 48 projects) to the pattern \verb!`\s*'! (found in X projects) when looking for lines devoid of content.  Without the endpoint anchors, the pattern matches every line, since there are always at least zero whitespace characters on every line.  But with the endpoint anchors, only lines that contain nothing but whitespace will match, allowing the user to find all lines that don't have any content.



\subsection{Opportunities for Future Work}



\subsubsection{New library feature for properties of a line}

\subsubsection{Regexes need refactoring}
We see the same features implemented many ways, and we don't know why. It might be that some methods of implementation are more understandable. However, if tool designers do not support that, refactoring may be needed.

example with anchors and {\tt .*} in the middle which could be replaced by a comma?

\subsubsection{Using 0-9 instead of \\d}
Out of the 9727 patterns acceptable to Rex, 1498 contained the range \verb•0-9• within a character class, even though this is exactly equivalent to using \verb•\d• within a character class (which appeared within a character class only 237 times).  Why are users specifying digits using the RNG feature when a default character class already exists?  Is it to aid readability, or does this represent an opportunity for refactoring?

\subsubsection{dot-star at the end - refactoring?}
One of the most ubiquitous sub-patterns, \verb!`.*'! appeared in 1937 of the 9727 patterns acceptable by Rex, and appeared within the last four characters in 1161 of the 9727 patterns.  Out of the top 30 clusters containing the ANY feature, 26 also had \verb!`.*'! within the last four characters.

One reasonable explanation for this tendency to put \verb!`.*'! at the end of a pattern is that users want to disregard all matches after the first match on a single line in order to count how many distinct lines the match occurs on, as illustrated in Figure~\ref{fig:lineSearch}.

\begin{figure}[htb]
\centering
\includegraphics[width=\columnwidth]{../illustrations/lineSearch.eps}
\caption{An Example of Using .* to Count Lines Containing `Apples'}
\label{fig:lineSearch}
\end{figure}

In many cases, this \verb!`.*'! at the end of the string may not actually contribute any new behavior to the pattern, and may in fact be extraneous.  Or users may be trying to bypass the whole-string nature of the {\tt re.match} function, without realizing that they could instead use the {\tt re.search} function.  Consider the comparison shown in Figure~\ref{fig:searchVSmatch}

\begin{figure}[htb]
\centering
\includegraphics[width=\columnwidth]{../illustrations/matchVSsearch.eps}
\caption{An Example of Using .* to get Search Behavior From The Match Function}
\label{fig:searchVSmatch}
\end{figure}

Are programmers using the dot-star sub-pattern unnecessarily? More research is needed into this question to find out if these patterns are a candidate for refactoring.






Fun fact: while creating similarity matrix, row 5464 took 2 hours, or almost 1 second per cell avg, only suffering 18 timeouts (1.2 secs).  What is this pesky pattern?

We do not assume that Python projects represent a perfect sample of regular expression usage in all environments, but to make the work of collecting data for the paper reasonable, we had to choose one language to focus on (we hope to compare results across languages in future work).  Python is an attractive choice because the culture of Python programming makes it seem likely that someone would write the pattern directly in the function, not trying to over-complicate things with some extra Classes or functions.  Other attractive choices are Perl (which probably has the most active regex community), javascript and ruby (which may emphasize web tasks like form validation), sql or a general purpose language like java or C\#.








