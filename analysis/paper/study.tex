\section{Study}
% \subsection{\centerline{DEFINITIONS}}

% \begin{center}
% \begin{tabular}
% {lcc}
% term & meaning & example  \\
% \toprule
% pattern & a composition of regex tokens & 'ab*c' \\
% \midrule
% usage & distinct function, pattern and flags within one filepath of one project & re.compile('ab*c') \\
% \midrule
% \end{tabular}
% \end{center}

\subsection{\centerline{RECORDING REGEX USAGES}}

We used the github api to page through all repositories, cloning projects that contain Python code, stopping when the scraper ran out of memory\footnote{\url{www.details.#better_parser}}.  For each project, we used Astroid[X] to build the AST of each Python file and find uses of Python's 're' module.  Here is an example of one regex \emph{usage}, with key components labeled:

\begin{figure}[htb]
\centering
\includegraphics[width=\columnwidth]{../illustrations/exampleUsage.eps}
\caption{example of one regex usage}
\label{fig:exampleUsage}
\end{figure}

Within each project, duplicate usages (same function, pattern and flags) within the same file (same relative path) were ignored.  Using git, each project was scanned at 20 evenly-spaced commits (or all commits if there were less than 20) in its history.  We observed and recorded \DTLfetch{data}{key}{nUsages}{value} regex usages in \DTLfetch{data}{key}{nProjScanned}{value} projects.

\subsection{\centerline{SELECTING PATTERNS}}
Our analysis focuses on the patterns found, so we ignored the \DTLfetch{data}{key}{percentBadFlags}{value}\%  of usages using flags that can alter regex behavior.  An additional \DTLfetch{data}{key}{percentInvalidPattern}{value}\% of usages contained patterns that could not be compiled because the pattern was non-static (used some runtime variable), or because of other unknown parsing failures.

The remaining \DTLfetch{data}{key}{percentCleanUsages}{value}\% (\DTLfetch{data}{key}{nCleanUsages}{value}) usages were collapsed into \DTLfetch{data}{key}{nDistinctPatterns}{value} distinct pattern strings.  The resulting set of patten strings were parsed using an antlr-based, open source PCRE parser released by Bart Kiers\footnote{\url{https://github.com/bkiers/pcre-parser}}.  This parser was unable to support \DTLfetch{data}{key}{percentUnicode}{value}\% (\DTLfetch{data}{key}{N_UNICODE}{value}) of the patterns due to unsupported unicode characters.  Another \DTLfetch{data}{key}{percentAlien}{value}\% (\DTLfetch{data}{key}{N_ALIEN}{value}) of the patterns used regex features that we have chosen to exclude in this study\footnote{\url{www.details.#thistopic}}.  The \DTLfetch{data}{key}{nCorpus}{value} distinct pattern strings that remain were each assigned a weight value equal to the number of distinct projects the pattern appeared in.  We will refer to this set of weighted, distinct pattern strings as the \emph{collection}.

\subsection{\centerline{ANALYZING FEATURES}}

After picking four large regex research projects, the big table with the features was created in order to decide which unsupported features are used most often.  Then our analysis breaks into two branches, basically the positive and negative perspective on feature usage.

For the negative perspective, we picked three features: LZY, NCG, WNW that are unsupported by Rex and other projects.  For each of these features, we created a subset of the \emph{collection} where all the patterns contain that feature.  Then we used syntactic analysis...to create a similarity matrix.  We then used markov clustering [X] (MCL) to find clusters in the subset.  We used these clusters to assist our manual search for some common use cases for the unsupported feature.

For the positive perspective, we created another subset of patterns (XYZ patterns) where Rex was able to generate strings that the pattern matched.  We then created a similarity graph with weighted, undirected edges as follows:
\begin{description}
\setlength{\parskip}{0pt} % block paragraphs
\setlength{\itemindent}{0in}
\item for each row i:
\setlength{\itemindent}{0.2in}
\item obtain set of Rex-generated strings Ri from pattern at index i
\item sRi = size of Ri
\item for each col j:
\setlength{\itemindent}{0.4in}
\item Nij = number of strings in Ri matched by pattern at index j
\item M[i][j] = Nij/sRi
\setlength{\itemindent}{0in}
\item G = empty graph
\item for each row i:
\setlength{\itemindent}{0.2in}
\item for each col j:
\setlength{\itemindent}{0.4in}
\item SIMij = (M[i][j]+M[j][i])/2
\item if SIMij > 0.75:
\setlength{\itemindent}{0.6in}
\item add edge (i,j)=SIMij to G
\setlength{\itemindent}{0in}
\setlength{\parskip}{10pt} % block paragraphs
\end{description}

Again we used MCL to find clusters that aided a manual search for use cases strongly associated with particular features.


TODO - list research questions (metrics and how they are computed?)

RQ1: How is the {\tt re module} used in python projects?
RQ2: How frequently are regexes used in python projects?
RQ3: Which regex language features are used most commonly in python?
RQ4: How syntactically diverse are regexes used in python?
RQ5: How semangically diverse are regexes used in python?

\subsubsection{\centerline{FEATURES}}

Here is a table showing all the features included in this study and which features are supported by four popular regex research projects/tools:

%\topcaption{This is top caption}
\bottomcaption{This is bottom caption}
\tablecaption{which features are studied, how this study names them, example project usage and }
\tablefirsthead{code & description & example & Rex & Hampi & RE2 & Brics & Weight & \% Weight \\ \toprule }
%
\tablehead{%
\multicolumn{2}{c}%
{{\bfseries  Continued from previous column}} \\
 \midrule code & description & example usage\\ \toprule}
%
\tabletail{%
\midrule \multicolumn{2}{r}{{Continued on next column}} \\ \midrule}
\tablelasttail{%
\bottomrule}

\begin{xtabular}{lllcccccc}
DEC &  any of: 0123456789 &
\begin{minipage}{0.5in}\begin{verbatim}
\d
\end{verbatim}\end{minipage} & Y & Y & Y & Y & 5,630 & 15.9\\
\midrule
NDEC & any non-decimal &
\begin{minipage}{0.5in}\begin{verbatim}
\D
\end{verbatim}\end{minipage} & Y & Y & Y & Y & 83 & 00.2\\
\midrule
WSP &
\textbackslash t \textbackslash n \textbackslash r \textbackslash b \textbackslash f or space
&
\begin{minipage}{0.5in}\begin{verbatim}
\s
\end{verbatim}\end{minipage} & Y & Y & Y & Y & 7,751 & 21.9\\
\midrule
NWSP & non-whitespace &
\begin{minipage}{0.5in}\begin{verbatim}
\S
\end{verbatim}\end{minipage} & Y & Y & Y & Y & 949 & 02.7\\
\midrule
WRD & letters, underscore, digits &
\begin{minipage}{0.5in}\begin{verbatim}
\w
\end{verbatim}\end{minipage} & Y & Y & Y & Y & 3,861 & 10.9\\
\midrule
NWRD & non-word chars &
\begin{minipage}{0.5in}\begin{verbatim}
\W
\end{verbatim}\end{minipage} & Y & Y & Y & Y & 288 & 00.8\\
\midrule
RNG & use of range shorthand &
\begin{minipage}{0.5in}\begin{verbatim}
[a-z]
\end{verbatim}\end{minipage} & Y & Y & Y & Y & 6,901 & 19.5\\
\midrule
LKA & matching sequence follows &
\begin{minipage}{0.5in}\begin{verbatim}
a(?=bc)
\end{verbatim}\end{minipage} & Y & Y & Y & Y & 401 & 01.1\\
\midrule
NLKA & sequence doesn't follow &
\begin{minipage}{0.5in}\begin{verbatim}
a(?!yz)
\end{verbatim}\end{minipage} & Y & Y & Y & Y & 582 & 01.6\\
\midrule
LKB & matching sequence precedes &
\begin{minipage}{0.5in}\begin{verbatim}
(?<=a)bc
\end{verbatim}\end{minipage} & Y & Y & Y & Y & 268 & 00.8\\
\midrule
NLKB & sequence doesn't precede &
\begin{minipage}{0.5in}\begin{verbatim}
(?<!x)yz
\end{verbatim}\end{minipage} & Y & Y & Y & Y & 355 & 01.0\\
\midrule
NCG & group without capturing &
\begin{minipage}{0.5in}\begin{verbatim}
a(?:b)c
\end{verbatim}\end{minipage} & Y & Y & Y & Y & 2,316 & 06.6\\
\midrule
CCC & custom char class &
\begin{minipage}{0.5in}\begin{verbatim}
[aeiou]
\end{verbatim}\end{minipage} & Y & Y & Y & Y & 11,558 & 32.7\\
\midrule
CG & capturing group &
\begin{minipage}{0.5in}\begin{verbatim}
(caught)
\end{verbatim}\end{minipage} & Y & Y & Y & Y & 16,890 & 47.8\\
\midrule
ANY & match any char &
\begin{minipage}{0.5in}\begin{verbatim}
.
\end{verbatim}\end{minipage} & Y & Y & Y & Y & 9,901 & 28.0\\
\midrule
LIT & a literal char &
\begin{minipage}{0.5in}\begin{verbatim}
a
\end{verbatim}\end{minipage} & Y & Y & Y & Y & 81\\
\midrule
NCCC & negated custom char class &
\begin{minipage}{0.5in}\begin{verbatim}
[^qxf]
\end{verbatim}\end{minipage} & Y & Y & Y & Y & 81\\
\midrule
BKR & match the $i^{th}$ capture  &
\begin{minipage}{0.5in}\begin{verbatim}
(.)\1
\end{verbatim}\end{minipage} & Y & Y & Y & Y & 81\\
\midrule
OR & logical or &
\begin{minipage}{0.5in}\begin{verbatim}
a|b
\end{verbatim}\end{minipage} & Y & Y & Y & Y & 81\\
\midrule
END & match end-of-line &
\begin{minipage}{0.5in}\begin{verbatim}
$
\end{verbatim}\end{minipage} & Y & Y & Y & Y & 81\\
\midrule
STR & match start-of-string &
\begin{minipage}{0.5in}\begin{verbatim}
^
\end{verbatim}\end{minipage} & Y & Y & Y & Y & 81\\
\midrule
WNW & word/non-word boundary &
\begin{minipage}{0.5in}\begin{verbatim}
\b
\end{verbatim}\end{minipage} & Y & Y & Y & Y & 81\\
\midrule
NWNW & negated WNW &
\begin{minipage}{0.5in}\begin{verbatim}
\B
\end{verbatim}\end{minipage} & Y & Y & Y & Y & 81\\
\midrule
ADD & one-or-more repetition &
\begin{minipage}{0.5in}\begin{verbatim}
z+
\end{verbatim}\end{minipage} & Y & Y & Y & Y & 81\\
\midrule
DBB & $n\le x \le m$ repetition &
\begin{minipage}{0.5in}\begin{verbatim}
z{3,8}
\end{verbatim}\end{minipage} & Y & Y & Y & Y & 81\\
\midrule
KLE & zero-or-more repetition &
\begin{minipage}{0.5in}\begin{verbatim}
.*
\end{verbatim}\end{minipage} & Y & Y & Y & Y & 81\\
\midrule
LWB & at least n repetitions &
\begin{minipage}{0.5in}\begin{verbatim}
z{15,}
\end{verbatim}\end{minipage} & Y & Y & Y & Y & 81\\
\midrule
QST & zero-or-one repetitions &
\begin{minipage}{0.5in}\begin{verbatim}
z?
\end{verbatim}\end{minipage} & Y & Y & Y & Y & 81\\
\midrule
SNG & exactly n repetitions &
\begin{minipage}{0.5in}\begin{verbatim}
z{8}
\end{verbatim}\end{minipage} & Y & Y & Y & Y & 81\\
\midrule
LZY & as few reps as possible &
\begin{minipage}{0.5in}\begin{verbatim}
z+?
\end{verbatim}\end{minipage} & Y & Y & Y & Y & 81\\
\midrule
NCG & named capture group &
\begin{minipage}{0.5in}\begin{verbatim}
(?P<name>x)
\end{verbatim}\end{minipage} & Y & Y & Y & Y & 81\\
\midrule
ENDZ & end of entire input &
\begin{minipage}{0.5in}\begin{verbatim}
end\Z
\end{verbatim}\end{minipage} & Y & Y & Y & Y & 81\\
\midrule
VWSP & vertical whitespace &
\begin{minipage}{0.5in}\begin{verbatim}
\v
\end{verbatim}\end{minipage} & Y & Y & Y & Y & 81\\
\midrule
OPT & inline flag options &
\begin{minipage}{0.5in}\begin{verbatim}
(?i)CasE
\end{verbatim}\end{minipage} & Y & Y & Y & Y & 81\\
\midrule
BKRN & named back-reference &
\begin{minipage}{0.5in}\begin{verbatim}
\g<name>
\end{verbatim}\end{minipage} & Y & Y & Y & Y & 81\\
\end{xtabular}










